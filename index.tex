% Created 2022-10-30 dom 22:04
% Intended LaTeX compiler: lualatex
\documentclass[letterpaper, 11pt]{article}
                      \usepackage{polyglossia}
\setmainlanguage{spanish}
\usepackage[citestyle=authoryear-icomp,bibstyle=authoryear,backend=biber,natbib=true]{biblatex}
\usepackage[utf8]{inputenc}
\usepackage{graphicx}
\usepackage{amsmath, amsthm, amssymb}
\usepackage[table, xcdraw]{xcolor}
\definecolor{bblue}{HTML}{0645AD}
\usepackage[colorlinks]{hyperref}
\hypersetup{colorlinks, linkcolor=blue, urlcolor=bblue}
\usepackage{titling}
\setlength{\droptitle}{-6em}
\setlength{\parindent}{0pt}
\setlength{\parskip}{1em}
\usepackage[stretch=10]{microtype}
\usepackage{hyphenat}
\usepackage{ragged2e}
\usepackage{subfig} % Subfigures (not needed in Org I think)
\usepackage{hyperref} % Links
\usepackage{listings} % Code highlighting
\usepackage[top=1in, bottom=1.25in, left=1.55in, right=1.55in]{geometry}
\renewcommand{\baselinestretch}{1.15}
\usepackage[explicit]{titlesec}
\pretitle{\begin{center}\fontsize{20pt}{20pt}\selectfont}
\posttitle{\par\end{center}}
\preauthor{\begin{center}\vspace{-6bp}\fontsize{14pt}{14pt}\selectfont}
\postauthor{\par\end{center}\vspace{-25bp}}
\predate{\begin{center}\fontsize{12pt}{12pt}\selectfont}
\postdate{\par\end{center}\vspace{0em}}
\titlespacing\section{0pt}{5pt}{5pt} % left margin, space before section header, space after section header
\titlespacing\subsection{0pt}{5pt}{-2pt} % left margin, space before subsection header, space after subsection header
\titlespacing\subsubsection{0pt}{5pt}{-2pt} % left margin, space before subsection header, space after subsection header
\usepackage{enumitem}
\setlist{itemsep=-2pt} % or \setlist{noitemsep} to leave space around whole list
\author{Lucas Fernandez}
\date{\today}
\title{Penitenciario}
\hypersetup{
 pdfauthor={Lucas Fernandez},
 pdftitle={Penitenciario},
 pdfkeywords={},
 pdfsubject={},
 pdfcreator={Emacs 28.2 (Org mode 9.5.5)}, 
 pdflang={English}}
\begin{document}

\maketitle
\tableofcontents

\section{Penitenciario}
\label{sec:org84b049b}

\subsection{Alternativas}
\label{sec:org558a193}

Respecto de la posibilidad de interrumpir una pena privativa de libertad y
remplazarla por el régimen de libertad vigilada intensiva, señale la alternativa
correcta:

I. La pena mixta, no es en verdad una pena, por lo mismo el tribunal para determinar
si procede no es el que condenó al postulante si no el de ejecución

II.Para acceder a la pena mixta es necesario que el penado haya cumplido al menos
un tercio de la pena privativa de libertad de manera efectiva

III.La sanción impuesta al condenado no puede ser mayor a cinco años
de presidio o reclusión menor en grado máximo y el postulante no debe
presentar condenas anteriores, aun cuando hayan sido impuestas bajo el
régimen de la Ley 20.084 de responsabilidad penal adolescente


a) Sólo I

b) I y II

c) Todas son correctas

d) II y III

e) Sólo III

\begin{itemize}
\item Respuesta:
\end{itemize}
b)

En nuestra normativa penitenciaria el fin atribuido a las penas privativas de
libertad durante la etapa de su ejecución es:

I. La prevención especial negativa

II. La retribución por el delito cometido

III. La prevención especial positiva

IV. La prevención general

a) Solo I

b) II y III

c) Solo III

d) Solo IV

e) Ninguna de las anteriores

\begin{itemize}
\item Respuesta:
\end{itemize}
c)


En relación con las personas condenadas presidio perpetuo simple, es correcto
señalar que:

I. Podrán postular a la salida dominical cuando hubieren cumplido 19 años de su
pena, siempre que concurran los demás requisitos establecidos en el Reglamento
de Establecimientos Penitenciarios.

II.Podrán acceder al beneficio de reducción de condena, contemplado en la Ley
19.856, pero sólo en la medida que se les haya reconocido en la respectiva
sentencia la circunstancia atenuante de la irreprochable conducta anterior.

III.Sólo pueden acceder a la salida esporádica.

a) Sólo I

b) I y II

c) Todas son correctas

d) Sólo II es correcta

e) I y III son correctas

\begin{itemize}
\item Respuesta:
\end{itemize}
a)


Iván Vásquez fue condenado en julio de 2011 a la pena de 10 años y 1
día de presidio mayor en su grado medio, por el delito de robo con
violencia simple.  Durante ese período Iván ha trabajado como mozo en
la cocina y ha realizado varios talleres en ese ámbito, su conducta es
muy buena desde hacer 3 años en que le sorprendieron 10 gramos. de
marihuana entre sus ropas. No asiste a la Escuela porque ya antes de
ser condenado, cursaba cuarto año de Derecho en una Universidad
privada. Considerando la reforma de la Ley 20.931 ¿cuáles de estas
afirmaciones son ciertas?

I.Ahora, el tiempo mínimo de postulación a la libertad condicional en
el caso del sr.  Vásquez es de 6 años y 8 meses y 1 día

II. La ley de “agenda corta” no afecta sus posibilidades de acceder al
beneficio de libertad condicional

III. Mientras no asista a la Escuela del penal no cumplirá el 4°
requisitos del artículo 2 del DL 321, por lo que no podrá acceder a la
libertad condicional

a) Sólo I

b) I y IV

c) Sólo II

d) Sólo III

e) III y IV

\begin{itemize}
\item Respuesta:
\end{itemize}
c)


\begin{enumerate}
\item Respecto del indulto particular, son correctas las siguientes afirmaciones:
\end{enumerate}

I.No procede respecto de todo delito

II.Consiste en la remisión, conmutación o reducción de la pena

III.Elimina la calidad de condenado para todos los efectos legales

IV.Exige el inicio de la condena en el establecimiento penal respectivo, en caso de
condena privativa de libertad efectiva

a) I, II y IV

b) II y III

c) II y IV

d) I, II y III

e) Todas son correctas

\begin{itemize}
\item Respuesta:
\end{itemize}
a)


En relación al beneficio de reducción de condena es correcto señalar lo
siguiente:

I.El beneficio de reducción de condena se concede por la “Comisión de beneficio
reducción de condena”

II.El condenado que ha cumplido las condiciones impuestas para acceder a la libertad
condicional está excluido del beneficio de reducción de condena

III.El beneficio de reducción de condena opera solo una vez en la vida de una
persona, de tal manera que si el condenado ya obtuvo este beneficio respecto de
una condena anterior queda excluido en relación con otras condenas futuras

a) Todas son correctas

b) Sólo I

c) I y II

d) Sólo III

e) II y III

\begin{itemize}
\item Respuesta:
\end{itemize}
d)



Los meses acumulados por rebaja de condena pueden perderse en caso de:

I.Incurrir el condenado en alguna de las causales de exclusión del artículo 17 de la
Ley de rebaja de condena

II.Cuando el condenado obtiene la libertad condicional

III.Cuando en un período de calificación, la Comisión de Rebaja de Condena califica el
comportamiento del condenado como “no sobresaliente”

IV.Nunca pueden perderse, es un derecho del condenado

a) Solo I

b) I y II

c) Solo III

d) Solo IV

e) I y III

\begin{itemize}
\item Respuesta:
\end{itemize}
e)


En relación con las normas del Reglamento de Establecimientos Penitenciarios
sobre las sanciones disciplinarias, señale la alternativa correcta:

I.Para la adopción de una sanción en forma ajustada a la falta, se considerarán,
además de la gravedad de la misma, la conducta del interno dentro del año

II.Según el Reglamento de Establecimientos Penitenciarios, en caso de faltas menos
graves o leves no es obligatorio que antes de aplicarse la sanción se escuche
personalmente al infractor

III.Sólo en caso de faltas graves debe remitirse una copia de la resolución que aplica
la sanción al Director Regional de Gendarmería, quien podrá anularla o modificarla
por razones fundadas

a) Todas son correctas

b) Ninguna es correcta

c) Sólo II es correcta

d) I y III son correctas

e) II y III son correctas

\begin{itemize}
\item Respuesta:
\end{itemize}
a)


El beneficio de reducción de condena consiste en:

I.La rebaja del tiempo de condena equivalente a dos meses por cada año de
cumplimiento en el cual el condenado hubiere demostrado comportamiento
sobresaliente.

II.A partir de la mitad de la condena la rebaja de tiempo se ampliará a
tres meses por cada año de comportamiento sobresaliente

III. La rebaja del tiempo de condena equivalente a tres meses por cada año de
cumplimiento en el cual el condenado hubiere demostrado comportamiento
sobresaliente

IV. La rebaja del tiempo de condena equivalente a 1 mes por cada año de
cumplimiento en el cual el condenado hubiere demostrado comportamiento
sobresaliente

a) Sólo I

b) Sólo II

c) Sólo III

d) Sólo IV

e) Ninguna de las anteriores

\begin{itemize}
\item Respuesta:
\end{itemize}
b)


En relación al Tribunal de Conducta, señale cuál o cuáles de las siguientes
afirmaciones es/son correctas:

I.Las notas para calificar la conducta de los reclusos son: pésima, mala, regular,
buena y muy buena

II.Sólo si el recluso ha experimentado durante el período de calificación un
comportamiento excepcionalmente sobresaliente podrá aumentar su calificación
de pésima a muy buena

III.Para la calificación de la conducta el Tribunal sólo puede tomar en consideración el
comportamiento que el recluso ha tenido en su módulo y patio.

IV.El tribunal de conducta puede estar integrado por civiles no funcionarios de
Gendarmería de Chile

V.El Tribunal de Conducta además, tiene por función determinar los postulantes a la
libertad condicional en el tiempo previo a cada período de funcionamiento de la
Comisión de Libertad condicional

a) Sólo V

b) I, IV y V

c) I, II y III

d) I y III

e) Todas las anteriores

\begin{itemize}
\item Respuesta:
\end{itemize}
b)

Obtenida la libertad condicional, el condenado quedará sujeto a las siguientes
obligaciones:

I.A presentarse a la unidad penal donde cumplía condena una vez al mes

II.Obedecer todas las órdenes legales que imparta el tribunal de conducta

III.No salir de su lugar de residencia sin autorización del Tribunal de conducta

a) II y III

b) I y II

c) I y III

d) Sólo III

e) Sólo I

\begin{itemize}
\item Respuesta:
\end{itemize}
a)


¿Cuál o cuáles de las siguientes afirmaciones son correctas respecto de la
sanción de internación en celda solitaria?:

I. Según el Reglamento de Establecimientos Penitenciarios, la aplicación de esta
sanción implica necesariamente una rebaja en la calificación de la conducta

II.No puede aplicarse esta sanción a las mujeres que tuvieren hijos consigo

III.Los internos sancionados con esta medida deberán ser visitados diariamente por el
Jefe del Establecimiento

a) Sólo III es correcta

b) I y II son correctas

c) I y III

d) II y III son correctas

e) Todas son correctas

\begin{itemize}
\item Respuesta:
\end{itemize}
e)


Las Reglas Mínimas de las Naciones Unidas para el tratamiento del delincuente
establecen:

I.Los objetivos de las penas y medidas privativas de libertad son principalmente
proteger a la sociedad contra el delito y reducir la reincidencia

II.Son muy antiguas y se encuentran desfasadas

III.Un sistema penitenciario modelo a ser seguido por las diversas legislaciones

IV.Que sus normas no se aplican a adolescentes

a) I, II y III

b) I y II

c) I y IV

d) II y III

e) Todas las anteriores

\begin{itemize}
\item Respuesta:
\end{itemize}
c)


Las siguientes afirmaciones son correctas en relación a la posición de garante
del Estado respecto de las personas privadas libertad:

I.Implica que Gendarmería debe garantizar aquellos derechos que los reclusos no
pueden satisfacerse a sí mismos, por su particular situación de dependencia en
relación con el órgano del Estado que ejerce su custodia

II.Gendarmería es responsable de las agresiones que sufra una persona en la cárcel,
sea provocada por funcionarios o producto de un riña con otros reclusos

III.La privación de libertad trae a menudo, como consecuencia ineludible, la
afectación del goce de otros derechos humanos además del derecho a la libertad
personal; pueden, por ejemplo, verse restringidos los derechos de privacidad y de
intimidad familiar. Esta restricción de derechos, consecuencia de la privación de
libertad o efecto colateral de la misma, sin embargo, debe limitarse de manera
rigurosa

a) Sólo I es correcta

b) I y II son correctas

c) I y III son correctas

d) II y III

e) Todas son correctas

\begin{itemize}
\item Respuesta:
\end{itemize}
e)

Las siguientes aseveraciones son correctas en relación con el abono de tiempo
de privación de libertad en causa diversa:

I.La jurisprudencia de la Corte Suprema ha señalado que al abono se aplica la regla
del artículo 164 del Código Orgánico de Tribunales, es decir, la exigencia de que
ambas causas (la de la privación de libertad que se abona y la de la condena en que
se abona el tiempo) hayan podido juzgarse conjuntamente

II.Sólo se puede abonar el tiempo de privación de libertad en causa diversa, cuando
en esta última se haya dictado sentencia absolutoria

III.El fundamento legal del abono se encuentra en el artículo 26 del Código Penal y no
son aplicables las reglas de la unificación de penas

a) Todas son correctas.

b) I y II son correctas.

c) Ninguna es correcta.

d) Sólo II es correcta.

e) I y III son correctas

\begin{itemize}
\item Respuesta:
\end{itemize}
e)


En relación al derecho a salud, son correctas las siguientes afirmaciones:

I.En caso de atenciones médicas que no puedan ser prestadas en el establecimiento
penal, el Director Regional podrá autorizar la internación en centros hospitalarios
externos

II.La atención médica al interior de la cárcel está limitada a las posibilidades
presupuestarias de Gendarmería, de modo que si no existe la posibilidad de
atención médica en determinadas especialidades, se la debe procurar el propio
condenado

III.La duración de la internación de los reclusos en establecimientos hospitalarios
externos será determinada por personal médico de Gendarmería de Chile

IV.En cada recinto penitenciario debe existir un módulo especial para las personas
que padezcan de enajenación mental

V.El condenado puede ser atendido en un establecimiento médico privado si cuenta
con los recursos para ello y ha sido autorizado por el Director Regional de
Gendarmería

a) Sólo I

b) Sólo II

c) III y IV

d) I, III y IV

e) II y IV

\begin{itemize}
\item Respuesta:
\end{itemize}
d)

Un condenado es sancionado a 10 días en internación en celda solitaria por
haber ocasionado lesiones a un compañero de celda. Como consecuencia de
ello, el jefe del establecimiento le señala que podrá salir de su celda de castigo
sólo una hora al día, que no podrá recibir encomiendas durante dicho período,
salvo artículos de higiene y limpieza, ni tampoco recibir visitas. ¿Cuál o cuáles
de las siguientes alternativas es correcta?

I.El jefe del establecimiento ha actuado dentro de sus atribuciones ya que es
facultativo de dicha autoridad imponer restricciones adicionales (visitas y
encomiendas) en el cumplimiento de la sanción de internación en celda solitaria

II.La aplicación de celda de castigo está prohibida en nuestro ordenamiento, por lo
que de ninguna manera se pudo haber impuesto esta medida

III.El jefe del establecimiento deberá poner en conocimiento del Ministerio Público lo
ocurrido para que se inicie la investigación correspondiente

IV.En caso que la causa iniciada por el delito de lesiones concluya en una sentencia
condenatoria, la sanción de aislamiento de 10 días impuesta al condenado deberá
imputarse a la eventual sanción de privación de libertad por el delito de lesiones ya
que nuestro ordenamiento penitenciario no permite la doble sanción administrativa y penal- por el mismo hecho

V.El jefe del establecimiento, al disponer la restricción de visitas, infringe la
prohibición de acumulación de sanciones

a) Sólo I

b) II y III

c) Sólo III

d) Sólo IV

e) III y V

\begin{itemize}
\item Respuesta:
\end{itemize}
e)


Francisco Pinto se encuentra condenado a 6 años de presidio mayor en su
grado mínimo por un delito de homicidio simple, solicita información acerca de
los requisitos para poder obtener salida dominical. El interno ha cumplido
exactamente 2 años de la pena impuesta, ha tenido muy buena conducta, ha
participado en el taller deportivo, pero no asiste a la escuela penitenciaria ya
que el curso que le corresponde (4° medio) no tiene vacantes. ¿Cuál
información proporcionada por el defensor(a) penitenciario(a) es correcta?:

I Usted deberá esperar 1 año más para poder postular a la salida dominical, ya que
le tiempo mínimo del homicidio son 2/3 de la pena. Además de mantener su buena
conducta y participación en las actividades que realiza

II. Usted deberá postular previamente a la salida esporádica y cumplir
adecuadamente durante al menos 6 meses

III.-Usted deberá esperar a tener un cupo en la escuela penitenciaria para poder
postular ya que uno de los requisitos exigidos por el reglamento de
establecimientos penitenciarios es asistir con regularidad y provecho a la escuela

IV.Usted está en condiciones de postular a la salida dominical, siempre que mantenga
su buena conducta y siga participando en las actividades que realiza

V.Usted deberá esperar 1 año y 6 meses más para poder postular a la salida
dominical y mantener su buena conducta y participación en las actividades que
realiza

a) Sólo I

b) Sólo II

c) Sólo III

d) Sólo IV

e) Sólo V

\begin{itemize}
\item Respuesta:
\end{itemize}
d)


En relación al beneficio de libertad condicional, es correcto afirmar:

I.En caso de presidio perpetuo calificado deberán transcurrir 40 años de
cumplimiento efectivo de la pena para poder postular a este beneficio

II.La concesión de la libertad condicional depende, en definitiva, del informe de
Gendarmería

III.Sólo podrán postular aquellos condenados que hayan registrado en conducta y
aplicación una nota equivalente a “muy bueno” durante el semestre anterior a la
postulación

IV.No pueden postular aquellos condenados que, teniendo buena conducta y tiempo
mínimo, no hayan participado en talleres


a) I, III y IV

b) I, II y IV

c) I, II y III

d) I y III

e) Todas las anteriores

\begin{itemize}
\item Respuesta:
\end{itemize}
d)

Señale la alternativa correcta en materia del llamado recurso de revisión:

I.No puede probarse por testigos los hechos en que se funda la solicitud de revisión

II.Por la acción de revisión se puede pedir la anulación de una sentencia absolutoria

III.Se puede revisar una sentencia condenatoria si ésta se ha fundado en un
testimonio, cuya falsedad ha sido declarada por un sentencia firme en causa
criminal

a) Todas son correctas

b) Sólo III

c) II y III

d) Sólo I es correcta

e) I y III

\begin{itemize}
\item Respuesta:
\end{itemize}
e)


Un condenado que acumula 7 años de condena por delitos de estafa, solicita
información acerca de los requisitos para poder obtener salida dominical. El
interno ha cumplido exactamente 2 años de la pena impuesta, ha tenido muy
buena conducta, ha participado en el taller deportivo, pero no asiste a la escuela
penitenciaria ya que el curso que le corresponde (2° medio) no tiene vacantes.
¿Cuál información proporcionada por el defensor(a) penitenciario(a) es correcta?:

I.-Usted deberá esperar 1 año y 6 meses más para poder postular a la salida
dominical y mantener su buena conducta y participación en las actividades que
realiza.

II.Usted deberá postular previamente a la salida esporádica y cumplir
adecuadamente durante al menos 6 meses.

III.-Usted deberá esperar a tener un cupo en la escuela penitenciaria para poder
postular ya que uno de los requisitos exigidos por el reglamento de
establecimientos penitenciarios es asistir con regularidad y provecho a la escuela.

IV.Usted está en condiciones de postular a la salida dominical.

V.-Usted deberá esperar 6 meses más para poder postular a la salida dominical y
mantener su buena conducta y participación en las actividades que realiza.

a)I y II
b)II y III
c)Ninguna es correcta
d)Sólo IV
e)Sólo V

\begin{itemize}
\item Respuesta:
\end{itemize}
d)

Respecto de la posibilidad de interrumpir una pena privativa de libertad y
remplazarla por el régimen de libertad vigilada intensiva, señale la alternativa
correcta:

I. Para acceder a la pena mixta es necesario que el penado haya cumplido al menos
la mitad de la pena privativa de libertad de manera efectiva.

II. La sanción impuesta al condenado no puede ser mayor a cinco años de presidio o
reclusión menor en su grado máximo.

III.De acuerdo a la Ley N°18.216, si se dispone la interrupción y reemplazo señalados,
la libertad vigilada intensiva deberá ser siempre controlada mediante monitoreo
telemático.	  

a) Todas son correctas.
b) I y II.
c) I y III.
d) II y III.
e) Sólo III.

\begin{itemize}
\item Respuesta:
\end{itemize}
e)

En relación con las personas condenadas como autores del delito de tráfico de
estupefacientes es correcto señalar que:

I. No podrán acceder al beneficio de reducción de condena, contemplado en la Ley
N° 19.856.

II. No podrán gozar de ningún permiso de salida, salvo el de salida esporádica.

III. No podrán postular al beneficio de libertad condicional.

a) Todas son correctas.
b) Ninguna es correcta.
c) Sólo I es correcta.
d) II y III son correctas.
e) Sólo III es correcta.

\begin{itemize}
\item Respuesta:
\end{itemize}
b)

Las siguientes aseveraciones son correctas en relación con el abono de tiempo de
privación de libertad en causa diversa:

I. La jurisprudencia de la Corte Suprema ha señalado en el último tiempo que al
abono se aplica la regla del artículo 164 del Código Orgánico de Tribunales (COT),
es decir, la exigencia de que ambas causas (la de la privación de libertad que se
abona y la de la condena en que se abona el tiempo) hayan podido juzgarse
conjuntamente.

II. Es procedente abonar a una condena el tiempo de prisión preventiva que sufrió
una persona en causa diversa cuando esta última causa terminó en sobreseimiento
definitivo.

III. Es procedente abonar a una condena el tiempo de privación de libertad que sufrió
una persona en causa diversa cuando esta última causa terminó por aplicación del
principio de oportunidad.

a) Todas son correctas.
b) I y II son correctas.
c) Ninguna es correcta.
d) Sólo II es correcta.
e) II y III son correctas.

\begin{itemize}
\item Respuesta:
\end{itemize}
a)

Luciano Vera fue condenado a una pena de 600 días de presidio menor en su grado
medio, sustituyéndosele por la pena sustitutiva de remisión condicional de la pena
por el lapso de 2 años. El Sr. Vera cometió un nuevo simple delito cuando había
cumplido 1 año de la mencionada pena sustitutiva, siendo condenado por
sentencia firme. ¿Cuáles de las siguientes afirmaciones es correcta en el caso del
Sr. Vera?

I.- El tiempo cumplido conforme a pena sustitutiva se puede abonar en caso de
revocación, por lo que a los 600 días de privación de libertad se deben restar los
365 días cumplidos bajo remisión condicional, lo que implica que le restan por
cumplir 235 días.

II.-No corresponde el abono y debe cumplir la pena privativa de libertad de manera
íntegra ya que se trata de modalidades de cumplimiento de naturaleza
incompatible.

III.Corresponde el abono pero éste debe hacerse de manera proporcional por lo que
le restan por cumplir 300 días de privación de libertad.

IV.-Cada día cumplido bajo la modalidad de remisión condicional equivale a un tercio
(8 horas) de un día de privación de libertad, por lo que los 365 días de pena
sustitutiva equivalen a 122 días de privación de libertad que deben restarse a los
600 originalmente impuestos, lo que significa que le restan 478 días de privación
de libertad.

a)Sólo I.

b)Sólo II.

c)Sólo III.

d)Sólo IV.

e)Ninguna es correcta.

\begin{itemize}
\item Respuesta:
\end{itemize}
c)



En cuanto a la aplicación del artículo 164 del Código Orgánico de Tribunales, que
regula la circunstancia de dictarse distintas sentencias condenatorias contra un
mismo imputado, es correcto señalar:

I. El tribunal competente es el tribunal que dictare el fallo posterior.

II. Según la jurisprudencia de la Corte Suprema, ambos procesos deben haber
coincidido en el tiempo, de modo que la sentencia del primer caso debe causar
ejecutoria después que se haya iniciado la investigación del segundo caso, y así
sucesivamente en caso de tratarse de varias condenas.

III. El objetivo de la norma es la unificación de penas, por lo que sólo opera en la
medida que pueda aplicarse el artículo 351 del Código Procesal Penal.

a) I y II son correctas.

b) II y III son correctas.

c) Todas son correctas.

d) Sólo II es correcta.

e) Sólo I es correcta.

\begin{itemize}
\item Respuesta:
\end{itemize}
a)

Señale cuál o cuáles de las siguiente afirmaciones son correctas respecto de los
permisos de salida contemplados en el Reglamento de establecimientos
penitenciarios:

I. La salida esporádica es el único permiso de salida que no se inspira en el carácter
progresivo del proceso de reinserción social.

II. Los condenados a penas inferiores a un año no pueden postular a los permisos de
salida.

III. La concesión, suspensión o revocación de los permisos de salida es una facultad
privativa del Jefe de Establecimiento.

a) Todas son correctas.

b) Sólo I y II son correctas.

c) Sólo II y III son correctas.

d) Sólo I y III son correctas.

e) Ninguna es correcta.

\begin{itemize}
\item Respuesta:
\end{itemize}
d)


En relación a la facultad de Gendarmería para trasladar reclusos, es acertado decir
que:

I. Sólo procede como sanción frente a una falta grave.

II. Debe ser autorizada por el juez de garantía del lugar de reclusión.

III. La decisión de traslado se materializa a través de un acto administrativo, por lo que
bien podría impugnarse si es que no se expresan claramente los fundamentos de
hecho del mismo.

a) Todas son correctas

b) Sólo I.

c) I y III.

d) II y III.

e) Sólo III.

\begin{itemize}
\item Respuesta:
\end{itemize}
e)

En relación al beneficio de reducción de condena es correcto señalar lo siguiente:

I. No procede este beneficio cuando en la condena se ha considerado concurrente
alguna de las circunstancias agravantes de los números 15 y 16 del artículo 12 del
Código Penal.

II. La reducción de condena se concede por decreto supremo, pero la autoridad
administrativa no puede revisar el mérito de lo obrado por la Comisión de
beneficio reducción de condena, salvo en cuanto a las causales de exclusión del
beneficio expresamente establecidas en el artículo 17 de la Ley N° 19.856.

III. El beneficio no es procedente respectos de condenados en libertad condicional.

a) Sólo I.

b) I y II.

c) Todas son correctas.

d) I y III.

e) Ninguna es correcta.

\begin{itemize}
\item Respuesta:
\end{itemize}
b)


En relación con el requisito relativo a la conducta exigido, entre otros, para acceder
a distintos beneficios, es correcto señalar lo siguiente:

I. Para acceder a salida dominical, de fin de semana y controlada al medio libre se
requiere haber observado muy buena conducta en los tres bimestres anteriores a
su postulación.

II. Para acceder a la libertad condicional se requiere haber observado conducta
intachable en el establecimiento penal, para lo cual el Tribunal de Conducta toma
en consideración las notas medias que tenga el interno en su Libro de Vida durante
el semestre anterior al primero de abril o primero de octubre de cada año,
respectivamente.

III.Para que un interno pueda participar del proceso de calificación de conducta
sobresaliente que realiza la “Comisión de beneficio reducción de condena”, se
requiere haber sido calificada su conducta con nota “buena” o “muy buena” en los
tres últimos bimestres anteriores al inicio del proceso de calificación.

a) Sólo I es correcta.

b) I y II son correctas.

c) Todas son correctas.

d) II y III son correctas.

e) I y III son correctas.

\begin{itemize}
\item Respuesta:
\end{itemize}
c)


Respecto de la pena mixta, es correcto señalar que:

I. No es procedente tratándose del autor de un homicidio calificado consumado
(artículo 391 N°1 del Código Penal), a menos que se le hubiera reconocido en la
sentencia la circunstancia primera del artículo 11 del Código Penal.

II. Sí es procedente si se trata del cómplice de homicidio calificado consumado.

III. No procede si se trata del cómplice del delito consumado de sustracción de menor
(artículo 142 del Código Penal).

a) Todas con correctas.

b) II y III son correctas

c) I y III son correctas.

d) I y II son correctas.

e) Sólo I es correcta.

\begin{itemize}
\item Respuesta:
\end{itemize}
d)


En relación con las normas relativas a las faltas disciplinarias, señale la alternativa
correcta:

I. Toda sanción será aplicada por el Jefe del Establecimiento.

II. En caso de falta menos grave no es obligatorio que antes de aplicarse la sanción el
Jefe del Establecimiento escuche personalmente al infractor.

III. La gradualidad de la rebaja de conducta consecuente de la aplicación de una
sanción por falta grave o menos grave la determinará el Tribunal de Conducta.

a) Todas son correctas.

b) Ninguna es correcta.

c) Sólo I es correcta.

d) I y III son correctas.

e) I y II son correctas.

\begin{itemize}
\item Respuesta:
\end{itemize}
a)

Son condiciones que debe cumplir la sanción de internación en celda solitaria:

I. Cuando se imponga por reiteración de infracciones menos graves, no puede
extenderse más allá de cinco días.

II. Copia de la resolución que impone esta sanción debe ser remitida al Director
Regional de Gendarmería para su conocimiento, quien podrá modificarla o anularla
por razones fundadas.

III. El interno afectado por esta medida debe ser visitado diariamente por el Jefe del
Establecimiento.

a) Sólo III es correcta.

b) Todas son correctas.

c) I y II son correctas.

d) I y III.

e) II y III son correctas.

\begin{itemize}
\item Respuesta:
\end{itemize}
e)


Señale la alternativa correcta en materia del llamado recurso de revisión:

I. La acción de revisión no procede en contra de sentencias condenatorias por faltas.

II. La acción de revisión no procede en contra de sentencias absolutorias.

III. La acción de revisión la conoce y falla el pleno de la Corte Suprema.

a) Sólo I es correcta.

b) I y II son correctas.

c) Todas son correctas.

d) Sólo II es correcta.

e) Ninguna es correcta.

\begin{itemize}
\item Respuesta:
\end{itemize}
b)


Se impone a un interno la sanción de privación de toda visita por un mes. ¿Cuál de
las siguientes afirmaciones es correcta de acuerdo a la normativa penitenciaria?

I. Si el condenado es reincidente es posible la acumulación de la sanción impuesta
con otra de menor entidad.

II. Esta sanción puede aplicarse a una infracción grave y, excepcionalmente, ante
reiteración de infracciones menos graves.

III. Si la infracción cometida pudiera constituir delito, la aplicación de la sanción debe
suspenderse hasta que se resuelva la situación por la Justicia Penal, de tal manera
que si se le condena judicialmente, la sanción administrativa quedará sin efecto.

a) Ninguna es correcta

b) Todas son correctas.

c) Sólo I es correcta.

d) Sólo II es correcta.

e) Sólo III es correcta.

\begin{itemize}
\item Respuesta:
\end{itemize}
a)

En relación con la libertad condicional es cierto que:

I. El beneficio se otorga por decreto supremo a propuesta de la Comisión de libertad
condicional.

II. Según el Reglamento de la Ley de Libertad Condicional, la persona que no sabe leer
y escribir, no cumple con uno de los requisitos para optar al beneficio.

III. Las personas condenadas a penas privativas de libertad inferiores a un año no
pueden optar al beneficio.

a) Sólo I.

b) Sólo II.

c) Sólo III.

d) II y III.

e) Todas son correctas.

\begin{itemize}
\item Respuesta:
\end{itemize}
d)


En relación con la regulación de las visitas que hace el Reglamento de
establecimientos penitenciarios, es correcto señalar:

I. Las visitas íntimas se conceden una vez al mes.

II. Las visitas especiales (familiares e íntimas) se conceden a internos que no gocen de
permisos de salida.

III. En las visitas ordinarias los menores de edad deberán tener más de catorce años.

a) Todas son correctas.

b) I y II.

c) I y III.

d) II y III.

e) Sólo II.

\begin{itemize}
\item Respuesta:
a)
\end{itemize}



Las siguientes afirmaciones son correctas en relación a la posición de garante del
Estado respecto de las personas privadas libertad:

I.La privación de libertad trae a menudo, como consecuencia ineludible, la
afectación del goce de otros derechos humanos además del derecho a la libertad
personal; pueden, por ejemplo, verse restringidos los derechos de privacidad y de
intimidad familiar. Esta restricción de derechos, consecuencia de la privación de
libertad o efecto colateral de la misma, sin embargo, debe limitarse de manera
rigurosa.

II.Gendarmería es responsable de las agresiones que sufra una persona en la cárcel,
sea provocada por funcionarios o producto de un riña con otros reclusos.

III.Los Estados excepcional y fundadamente pueden invocar privaciones económicas
para justificar condiciones de detención que no cumplen con los estándares
mínimos internacionales.

a) Sólo I es correcta.

b) Todas son correctas.

c) I y II son correctas.

d) I y III son correctas.

e) II y III.

\begin{itemize}
\item Respuesta:
\end{itemize}
c)


Sobre la pena sustitutiva de expulsión, son correctas las siguientes aseveraciones:

I.Según la Ley N° 18.216, el tribunal no puede de oficio sustituir el cumplimiento de
la pena impuesta por la expulsión del extranjero condenado.

II.El extranjero al que se le aplica la pena sustitutiva de expulsión no puede regresar
al territorio nacional en un plazo de diez años.

III.La imposición de la pena sustitutiva de expulsión a un extranjero que no hubiera
sido condenado anteriormente por crimen o simple delito tendrá mérito suficiente
para la omisión, en los certificados de antecedentes, de las anotaciones a que
diere lugar la sentencia condenatoria.


a) Todas son correctas.

b) I, II son correctas.

c) Sólo II es correcta.

d) II y III son correctas.

e) Sólo la III es correcta.

\begin{itemize}
\item Respuesta:
\end{itemize}
d)


En relación con el indulto particular es correcto señalar:

I.El indultado continúa con el carácter de condenado para los efectos de la
reincidencia.

II.La calificación de la concurrencia de los requisitos que permiten denegar el indulto
corresponde al Consejo Técnico respectivo.

III.El indulto puede consistir en la remisión, conmutación o reducción de la pena..

a) Todas son correctas.

b) Sólo I es correcta.

c) I y II son correctas.

d) II y III son correctas.

e) I y III son correctas.

\begin{itemize}
\item Respuesta:
\end{itemize}
e)


En relación con la salida dominical es correcta señalar:


I.Se requiere informe favorable del Consejo Técnico para su concesión.


II.Se requiere que la persona condenada haya observado muy buena
conducta

en los tres bimestres anteriores a su postulación.


III.Para postular se requiere que previamente al interno se le hay
concedido la salida esporádica y haya gozado provechosamente de la
misma.



a) Sólo I es correcta

b) I y II son correctas

c) Todas son correctas

d) I y III son correctas

e) II y III son correctas


\begin{itemize}
\item Respuesta:
\end{itemize}

b)


Son causales de exclusión del beneficio de reducción de condena:


I.- Haber ejecutado el delito por el que se cumple la condena en
desprecio o con ofensa de la autoridad pública o en el lugar que se
halle ejerciendo funciones.



II.- Que la condena hubiere sido dictada considerando concurrente
alguna de las circunstancias agravantes establecidas en los números 15
y 16 del artículo 12 del Código Penal.



III.- Haber sido condenado por el delito de violación de persona menor
de 14 años de edad.



a) Todas son correctas

b) I y II son correctas

c) II y III son correctas

d) Sólo II es correcta

e) Sólo III es correcta


\begin{itemize}
\item Respuesta:
\end{itemize}

d), hoy en día es correcta la c) dado que se incluyó la violación
impropia en el listado de restricciones a dicho beneficio. En relación
con la posibilidad de sustituir la pena impuesta a un extranjero por
su expulsión del territorio nacional, de acuerdo a lo dispuesto en la
Ley.216, es correcto señalar que:



I.La decisión que conceda o deniegue esta pena sustitutiva es apelable



II.Sólo procede respecto de una pena igual o inferior a cinco años de
presidio o reclusión menor en su grado máximo.



III.El condenado extranjero al que se le aplicare la pena de expulsión
no podrá regresar al territorio nacional en un plazo de diez años, contado
desde la fecha de la sustitución de la pena.





a) Todas son correctas

b) I y II son correctas

c) I y III son correctas

d) II y III son correctas

e) Sólo II es correcta


\begin{itemize}
\item Respuesta:
\end{itemize}

a)



En relación con el beneficio de reducción de condena es correcto
señalar que:


I.El órgano encargado de efectuar la calificación de comportamiento
necesaria para acceder al beneficio de reducción de condena es el
respectivo Tribunal de Conducta.



II.El órgano que concede el beneficio de reducción de condena es la
Comisión de beneficio de reducción de condena.



III.No procede este beneficio respecto de los condenados en libertad
condicional.




a) Todas son correctas

b) Ninguna es correcta

c) Sólo II es correcta

d) I y III son correctas

e) Sólo III es correcta


\begin{itemize}
\item Respuesta:
\end{itemize}

b), el organo calificador de la confducta es la Comisión de Beneficio
  de Reduicción de CondenaRespecto al número III, no es óbice la que
  el condenado esté en libertad condicional, dado que solamente es
  límite para dicho benefício que: a) incumpla las condiciones de la
  LC y b) deliquiere en cumplimiento de libertad condicional.


Señale la alternativa correcta en materia del llamado recurso de
revisión:


I.Por la acción de revisión se puede pedir la imposición de una pena
más favorable que la impuesta por la sentencia firme que se revisa
cuando esta última haya sido aplicada en base a antecedentes falsos.



II.No procede respecto de sentencias condenatorias por faltas.



III.No puede probarse por testigos los hechos en que se funda la
solicitud de revisión.


a) Todas son correctas

b) Sólo II es correcta

c) I y II son correctas

d) I y III son correctas

e) II y III son correctas


\begin{itemize}
\item Respuesta:

e) se busca la nulidad de la sentencia condenatoria por un crimen o
simple delito, quedando excluidas las faltas.
\end{itemize}



Respecto del indulto particular, son correctas las siguientes
afirmaciones:


I.El indulto extingue la responsabilidad penal.



II.-En casos calificados y mediante decreto supremo fundado, el
Presidente de la República podrá prescindir de los requisitos
establecidos en la Ley N° .050, sobre indultos particulares y de los
trámites indicados en su reglamento, siempre que el beneficiado esté
condenado por sentencia ejecutoriada y no se trate de conductas
terroristas, calificadas como tales por una ley dictada de acuerdo al
artículo 9° de la Constitución Política.



III.-No está permitido solicitar el indulto a las personas condenadas
por delitos violación o abuso sexual cometidos en contra de menores de
14 años.


a) Sólo I es correcta

b) Sólo II es correcta

c) Sólo III es correcta

d) I y II son correctas

e) Todas son correctas


\begin{itemize}
\item Respuesta:

d)
\end{itemize}



Respecto de la posibilidad de interrumpir una pena privativa de
libertad y remplazarla por el régimen de libertad vigilada intensiva,
señale la alternativa correcta:


I.De acuerdo a la Ley 18.216, en el caso que el tribunal dispusiere la
interrupción de la pena privativa de libertad, reemplazándola por el
régimen de libertad vigilada intensiva, ésta deberá ser siempre
controlada mediante monitoreo telemático.



II.Para acceder a la pena mixta se requiere haber cumplido al menos
la mitad de la condena.

III.Para acceder a la pena mixta se requiere estar gozando y hacer uso
provechoso de un permiso de salida.


a) I y III son correctas

b) I y II son correctas

c) II y III son correctas

d) Todas son correctas

e) Sólo I es correcta


\begin{itemize}
\item Respuesta:
\end{itemize}

e)



En relación a personas condenadas a presidio perpetuo calificado es
correcto afirmar:


I.No pueden postular a permisos de salida salvo que se trate de
salidas esporádicas.

II.Podrán postular al beneficio de libertad condicional transcurridos
20 años de cumplimiento efectivo de la pena.

III.No podrán acceder al beneficio de reducción de condena,
contemplado en la Ley N° 19.856.


a) Ninguna es correcta

b) Todas son correctas

c) Sólo III es correcta

d) II y III son correctas

e) I y III son correctas


\begin{itemize}
\item Respuesta:
\end{itemize}

c)


En materia de control judicial de la ejecución de la pena, es correcto
señalar que:


I.La regla general es que la competencia en materia de ejecución de
sentencias criminales y medidas de seguridad está radicada en el Juez
de Garantía del lugar en que la pena deba cumplirse.

II.El incumplimiento y el quebrantamiento de las penas sustitutivas le
corresponde conocerlo al tribunal que dictó la respectiva sentencia.


III.El tribunal competente para la llamada “unificación de penas” del
artículo 4 del Código Orgánico de Tribunales es el que dictó el último
de los fallos.


a) Todas son correctas

b) I y III son correctas

c) II y III son correctas

d) Sólo III es correcta

e) Sólo II es correcta


\begin{itemize}
\item Respuesta:
\end{itemize}

d)


Señale cuál o cuáles de las siguiente afirmaciones son correctas
respecto de los permisos de salida contemplados en el Reglamento de
establecimientos penitenciarios:


I.La salida esporádica sólo puede autorizarse con custodia o
vigilancia, independientemente de si el motivo de la misma consiste en
hechos de especial trascendencia familiar, comparecencia personal en
diligencias urgentes, premio o estímulo especial o actividades de
reinserción social.

II.En caso de quebrantamiento o incumplimiento voluntario de las
condiciones de algún permiso, al reingreso, el interno tendrá una
conducta calificada con la nota mínima.

III.Los condenados a penas inferiores a un año no pueden postular a
los permisos de salida.


a) Todas son correctas

b) I y II son correctas

c) II y III son correctas

d) Sólo II es correcta

e) Sólo I es correcta


\begin{itemize}
\item Respuesta:
\end{itemize}

b), la número III es erronea dado que el mismo art115 señala
expresamente que los condenados a pena inferior a 1 año podrán
postular a los permisos de salida cumpliendo los requisitos generales.

En relación con las normas del Reglamento de Establecimientos
Penitenciarios sobre las sanciones disciplinarias, señale la
alternativa correcta:


I.Toda sanción disciplinaria debe ser impuesta por el Tribunal de
Conducta.


II.La comisión de tres faltas menos graves durante un bimestre,
constituye una falta grave.

III.El Consejo Técnico constituye segunda instancia para que los
internos reclamen sobre la aplicación de una sanción disciplinaria.


a) Todas son correctas

b) Sólo I es correcta

c) Sólo II es correcta

d) Sólo III es correcta

e) I y II son correctas


\begin{itemize}
\item Respuesta:
\end{itemize}

c)


En relación con la regulación de las visitas que hace el Reglamento de
Establecimientos Penitenciarios, es correcto señalar:


I.Los Alcaides pueden autorizar visitas íntimas pero sólo a los
internos que no gocen de permisos de salida.



II.Los Jefes de los establecimientos pueden impedir las visitas de
personas cuya presentación sea indecorosa.



III.Para tener derecho a visitas es necesario haber sido evaluado en
los últimos bimestres con muy buena conducta.



a) Sólo I es correcta

b) I y II son correctas

c) II y III son correctas

d) Sólo III es correcta

e) Todas son correctas


\begin{itemize}
\item Respuesta:
\end{itemize}

b)



En relación con la pena mixta es correcto afirmar que:


I.El tribunal puede de oficio, siempre que concurran los requisitos
legales, disponer la interrupción de la pena privativa de libertad
originalmente impuesta, reemplazándola por el régimen de libertad
vigilada intensiva.



II.Una vez cumplida la mitad del período de observación, previo
informe favorable de Gendarmería de Chile, el tribunal podrá
reemplazar el régimen de libertad vigilada intensiva por el de
libertad vigilada.



III.Si el tribunal no otorga la pena mixta, esta no podrá discutirse
nuevamente sino hasta transcurridos seis meses desde de su denegación.




a) Todas son correctas

b) I y II son correctas

c) I y III son correctas

d) Sólo I es correcta

e) II y III son correctas


\begin{itemize}
\item Respuesta:
\end{itemize}

c)



Señale la o las afirmaciones correctas en materia de calificación de
conducta:


I.En cada bimestre el Tribunal de Conducta sólo podrá aumentar en un
grado la nota de conducta que haya obtenido un interno en el bimestre
anterior.



II.Cuando por la imposición de sanciones graves procede la rebaja en
la calificación de conducta, esta rebaja puede ser uno o más grados.



III.El Consejo Técnico constituye segunda instancia para que los
internos reclamen por la calificación de su conducta.



a) I y II son correctas

b) I y III son correctas

c) II y III son correctas

d) Todas son correctas

e) Sólo II es correcta


\begin{itemize}
\item Respuesta:
\end{itemize}

a)



Según el reglamento de Establecimiento Penitenciarios, en materia de
visitas a los internos, es correcto señalar lo siguiente:


I.Las visitas extraordinarias se pueden permitir por un lapso no
superior a media hora.



II.Las visitas familiares especiales sólo pueden concederse a los
internos que no gocen de permisos de salida.



III.Las visitas especiales (familiares o íntimas) no pueden exceder
las tres horas cada vez.




a) Todas son correctas

b) Ninguna es correcta

c) Sólo II es correcta

d) II y III son correctas

e) Sólo III es correcta


\begin{itemize}
\item Respuesta:
\end{itemize}

a)


¿Cuál o cuáles de las siguientes afirmaciones son correctas respecto
de las faltas disciplinarias?

I.La comisión de 3 faltas leves en un bimestre constituye una falta
menos grave.



II.La comisión de falta disciplinaria que pudiere constituir delito,
será puesta en conocimiento de la justicia penal, sin perjuicio de la
aplicación de las sanciones previstas en el Reglamento de
Establecimientos Penitenciarios.



III.Cuando se comete una falta grave, se puede imponer más de una de
las sanciones previstas en el Reglamento de Establecimientos
Penitenciarios, pero sólo en caso de reincidencia en el bimestre
respectivo.




a) Sólo I es correcta

b) Sólo II es correcta

c) I y II son correctas

d) I y III son correctas

e) Todas son correctas


\begin{itemize}
\item Respuesta:
\end{itemize}

c)


Uno de los requisitos que deben reunir los condenados para postular a
casi todos los permisos de salida es poseer “Muy buena Conducta”Esta
calificación se le exigirá por el lapso inmediatamente anterior a la
postulación de:


I.Dos bimestres.


II.Tres bimestres.


III.Cuatro bimestres.


IV.Últimos 12 meses.


a) Sólo I

b) Sólo II

c) Sólo III

d) Sólo IV

e) Ninguna es correcta


\begin{itemize}
\item Respuesta:
\end{itemize}

b)


Las siguientes aseveraciones son correctas en relación con el abono de
tiempo de privación de libertad en causa diversa:


I.Se puede abonar el tiempo de privación de libertad en causa diversa,
cuando en esta última se haya dictado sentencia absolutoria.



II.Se puede abonar el tiempo de privación de libertad en causa
diversa, cuando en esta última se haya aplicado el principio de
oportunidad.



III.Se puede abonar el tiempo de privación de libertad en causa
diversa, cuando en esta última se hay dictado sobreseimiento
definitivo.



IV.Procede el abono en causa diversa aunque esta última se haya
juzgado según el antiguo sistema procesal penal.



a) Sólo I es correcta

b) I y II son correctas

c) I y III son correctas

d) I y IV son correctas

e) Todas son correctas


\begin{itemize}
\item Respuesta:
\end{itemize}

e)


En relación con la sanción de internación en celda solitaria ¿Cuál o
cuáles de las siguientes alternativas es correcta?

I.La aplicación de esta sanción implica necesariamente una rebaja en
la calificación de la conducta en uno o más grados.



IICopia de la resolución que la impone debe ser remitida al Director
Regional de Gendarmería, quien podrá modificarla o anularla por
razones fundadas

IIIAntes de aplicarse la sanción, el Jefe del Establecimiento deberá
escuchar personalmente al infractor.


a) Todas son correctas

b) I y II son correctas

c) II y III son correctas

d) I y III son correctas

e) Sólo I es correcta



\begin{itemize}
\item Respuesta:
\end{itemize}

a)


Son requisitos para obtener la libertad condicional:


I.Haber cumplido el tiempo mínimo de la condena impuesta, el que se
establece en la ley según las diferentes hipótesis que regula.



II.Haber sido beneficiado previamente con permisos de salida.


III.Haber asistido con regularidad y provecho a la escuela del
establecimiento y a las conferencias educativas que se dicten,
entendiéndose que no reúne este requisito el que no sepa leer y
escribir.



a) Todas son correctas

b) I y II son correctas

c) Sólo I es correcta

d) Sólo II es correcta

e) I y III son correctas


\begin{itemize}
\item Respuesta:
\end{itemize}

e)



La libertad condicional, salvo el caso de los condenados a presidio
perpetuo calificado, es otorgada por:


a) El Tribunal de Conducta.


b) La Comisión de libertad condicional.


c) Decreto Supremo del Ministerio de Justicia.


d) El Alcaide del establecimiento penal a propuesta del Tribunal de
Conducta.


e) Ninguna de las anteriores es correcta.


\begin{itemize}
\item Respuesta:

b)
\end{itemize}


En relación al indulto particular, es correcto afirmar:

I No procederá en caso de condenas por delitos de tráfico de
estupefacientes.

II Quita al condenado su calidad de tal para todos los efectos
legales.

III Consiste en la conmutación, remisión o reducción de la pena.


a) Todas son correctas

b) Sólo I

c) I y II

d) Sólo II

e) Sólo III

\begin{itemize}
\item Respuesta:

e)
\end{itemize}


Un condenado a 9 años de presidio mayor en su grado mínimo, por el
delito de parricidio, solicita información acerca de los requisitos
para poder obtener salida dominicalEl interno ha cumplido exactamente
3 años de la pena impuesta, ha tenido muy buena conducta, ha
participado en el taller deportivo, pero no asiste a la escuela
penitenciaria ya que el curso que le corresponde (2° medio) no tiene
vacantes¿Cuál información proporcionada por el defensor(a)
penitenciario(a) es correcta?:


a) Usted deberá esperar 2 años más para poder postular a la salida
dominical y mantener su buena conducta y participación en las
actividades que realiza.

b) Usted no puede postular a ningún permiso de salida, salvo en caso
de muerte, enfermedad grave o accidente de su cónyuge o pariente
cercano, por encontrarse condenado por el delito de parricidio.

c) Usted deberá esperar a tener un cupo en la escuela penitenciaria
para poder postular ya que uno de los requisitos exigidos por el
reglamento de establecimientos penitenciarios es asistir con
regularidad y provecho a la escuela.

d) Usted deberá esperar 3 años más para poder postular a la salida
dominical y mantener su buena conducta y participación en las
actividades que realiza.

e) Usted deberá esperar 1 año más para poder postular a la salida
dominical y mantener su buena conducta y participación en las
actividades que realiza.

\begin{itemize}
\item Respuesta:

a)
\end{itemize}


Según el reglamento de establecimientos penitenciarios,
excepcionalmente el Director regional de Gendarmería que corresponda,
puede autorizar la internación de penados en establecimientos
hospitalarios externosSon afirmaciones correctas a este respecto:

I Dicha autorización procede en casos graves que requieran con
urgencia de atención o cuidados médicos especializados que no se pueda
otorgar en la unidad médica del establecimiento.

II También procede cuando el penado requiera atenciones médicas que,
sin revestir caracteres de gravedad o urgencia, no puedan ser
prestadas en el establecimiento.

III Incluso se puede autorizar que el interno sea atendido en el
establecimiento privado que desee, si es que cuenta con recursos para
financiar dicha atención.

IV La duración de la internación de los penados en recintos
hospitalarios externos, será determinada por el personal médico de
Gendarmería de Chile.


a) Sólo I

b) I y II

c) I, II y II

d) I, II y IV

e) Todas son correctas

\begin{itemize}
\item Respuesta:

e)
\end{itemize}


Respecto de la posibilidad de interrumpir una pena privativa de
libertad y remplazarla por el régimen de libertad vigilada intensiva,
señale la alternativa correcta:

I La aplicación de la llamada pena mixta requiere que la pena que se
está cumpliendo no supere los cinco años de privación de libertad.

II Para acceder a la pena mixta es necesario que el penado haya
cumplido al menos la mitad de la pena privativa de libertad de manera
efectiva.

III Si el tribunal no otorga la interrupción de la pena privativa de
libertad, no podrá a volver a discutirse el asunto sino hasta que
transcurran tres meses desde su denegación.


a) Sólo I

b) I y II

c) I, II y III

d) Sólo II

e) Ninguna es correcta

\begin{itemize}
\item Respuesta:

e), esto porque incluso la I está mal redactada, en el fondo se
puede solicitar y otorgar respecto a aquellos condenados a presidio
mayor en su grado mínimoEs decir con un tope máximo de 5 y 1 día.
\end{itemize}


En relación a personas condenadas a presidio perpetuo calificado es
correcto afirmar:

I No podrán acceder al beneficio de rebaja de condena, contemplado en
la Ley N° 19.856.

II No podrán gozar de ningún permiso de salida.

III Podrán postular al beneficio de libertad condicional transcurridos
20 años de cumplimiento efectivo de la pena.

a) Sólo I.


b) Sólo II.



c) I y II.


d) Sólo III.



e) I, II y III.


\begin{itemize}
\item Respuesta:

a)
\end{itemize}



“Bonifacio” se encuentra cumpliendo las siguientes penas que le fueron
impuestas en diversas sentencias: a) tres años y un día de presidio
menor en su grado máximo como autor del delito de robo con fuerza en
las cosas del art442 CP, sentencia dictada por juzgado del crimen; b)
siete años y 183 días de presidio mayor en su grado mínimo como autor
del delito de robo con intimidación (art.436 CP), impuesta por un TOP;
y c) dos años de presidio menor en su grado medio como autor del
delito de robo por sorpresa (art436 inciso segundo CP), impuesta por
un juzgado de garantía“Bonifacio” solicita al defensor penitenciario
que lo asesore para efectos de una posible aplicación del Art.164 del
Código Orgánico de Tribunales.

¿Cuál de las siguientes alternativas es la correcta?

a) No corresponde aplicar el art164 del COT porque es improcedente ya
que se trata de sentencias dictadas en diversos sistemas procesales.

b) Aunque las sentencias hubieran sido dictadas en un mismo sistema
procesal, no procede aplicar el art164 del COT ya que los delitos por
los que fue condenado no son de la misma especie.

c) Sólo se puede aplicar el art164 del COT respecto de los dos delitos
dictados en el nuevo sistema procesal penal.

d) Es procedente la aplicación del art.164 del COT si se puede estimar
que los delitos pudieron juzgarse conjuntamente.

e) Ninguna es correcta.


\begin{itemize}
\item Respuesta:

d)
\end{itemize}


Los permisos de salida contemplados en el Reglamento de
establecimientos penitenciarios (Decreto Supremo N° 518 del Ministerio
de Justicia), que le permiten a condenados a penas privativas de
libertad abandonar el establecimiento penitenciario, son (señale la
alternativa completamente correcta):



a) Reclusión parcial, libertad vigilada, libertad vigilada intensiva y
prestación de servicios en beneficio de la comunidad.

b) Salida dominical, salida controlada al medio libre y libertad
condicional.


c) Salidas esporádicas, salida dominical, salida de fin de semana y
salida controlada al medio libre.

d) Libertad condicional, reclusión parcial, libertad vigilada,
libertad vigilada intensiva y prestación de servicios en beneficio de
la comunidad.

e) Ninguna de las anteriores.


\begin{itemize}
\item Respuesta:

c)
\end{itemize}


Para la aplicación del régimen de extrema seguridad regulado en el
artículo 28 del reglamento de establecimientos penitenciarios se
requiere:


I Resolución fundada del Director Nacional de Gendarmería o de los
Directores Regionales en caso de delegación de funciones.

II Que el condenado lo sea por delitos de cierta gravedad.

III Informe técnico que recomiende la aplicación de este régimen.

IV La existencia de una situación que amenace la vida e integridad
física o psíquica de las personas en la unidad penal respectiva y/o el
orden y la seguridad del mismo.


a) I y IV

b) I, III y IV

c) III y IV

d) I, II y IV

e) Todas las anteriores

\begin{itemize}
\item Respuesta:

b), no se requiere de informe técnico alguno para la aplicación de
este régimen.
\end{itemize}


Uno de los requisitos que deben reunir los condenados para postular a
casi todos los permisos de salida es poseer “Muy buena Conducta”Esta
calificación se le exigirá por el lapso inmediatamente anterior a la
postulación de:


a) Un bimestre

b) Dos bimestres

c) Tres bimestres

d) Último semestre

e) Últimos 12 meses

\begin{itemize}
\item Respuesta:

c)
\end{itemize}


En relación al beneficio de reducción de condena es correcto señalar
lo siguiente:

I Se puede otorgar a una persona que está cumpliendo efectivamente una
pena privativa de libertad.

II Se puede otorgar a los condenados que cumplieren pena bajo
reclusión nocturna.

III El tiempo que un condenado hubiere permanecido en prisión
preventiva durante todo o parte del respectivo proceso, no se
computará para los efectos de la calificación de la conducta.

IV El órgano encargado de efectuar la calificación de comportamiento
necesaria para acceder a la reducción de condena es el Tribunal de
Conducta.


a) I y II

b) I, II y III

c) I, II y IV

d) I, III y IV

e) Sólo I

\begin{itemize}
\item Respuesta:

a), el organo calificador se denomina Comisión de Beneficio de
Reducción de Condena y éste es quien está a cargo de la
calificación.
\end{itemize}


En relación al trabajo penitenciario de los internos, es correcto
señalar:

I Es posible autorizar a internos para que realicen trabajos fuera del
recinto penitenciario.

II Si el trabajo se realiza fuera del establecimiento penitenciario,
sólo podrá ser en beneficio de la comunidad o encontrarse justificado
en relación a algún programa de rehabilitación, capacitación o empleo.

III La custodia y distribución de las remuneraciones que perciban los
internos corresponde al Jefe del Establecimiento.


a) Sólo I

b) I y II

c) Sólo III

d) Todas son correctas

e) I y III

\begin{itemize}
\item Respuesta:

d)
\end{itemize}


En materia penitenciaria, desde la perspectiva del Derecho
Internacional de los Derechos Humanos, es correcto señalar que:

I Los Estados excepcional y fundadamente pueden invocar privaciones
económicas para justificar condiciones de detención que no cumplen con
los estándares mínimos internacionales.

II La privación de libertad trae a menudo, como consecuencia
ineludible, la afectación del goce de otros derechos humanos además
del derecho a la libertad personal, Pueden, por ejemplo, verse
restringidos los derechos de privacidad y de intimidad familiarEsta
restricción de derechos, consecuencia de la privación de libertad o
efecto colateral de la misma, sin embargo, debe limitarse de manera
rigurosa.

III El Estado se encuentra en una posición especial de garante frente
a las personas privadas de libertad, toda vez que las autoridades
penitenciarias ejercen un fuerte control o dominio sobre las personas
que se encuentran sujetas a su custodia.


a) Sólo III

b) II y III

c) Todas son correctas

d) I y III

e) Ninguna es correcta

\begin{itemize}
\item Respuesta:

b)
\end{itemize}


En relación con las faltas disciplinarias señale la alternativa
correcta:

I La aplicación de toda sanción correspondiente a faltas graves o
menos graves, implica necesariamente una rebaja en la calificación de
la conducta.

II En caso de infracción grave y antes de aplicarse la sanción el Jefe
del establecimiento deberá escuchar personalmente al infractor.

III Toda sanción será aplicada por el Jefe del Establecimiento.


a) Todas son correctas

b) II y III

c) Sólo II

d) I y II

e) Sólo III

\begin{itemize}
\item Respuesta:

a)
\end{itemize}


Son condiciones que debe cumplir la sanción de internación en celda
solitaria:

I Debe cumplirse en la misma celda o en otra de análogas condiciones
de higiene, iluminación y ventilación.

II El interno afectado por esta medida no puede recibir paquetes,
salvo artículos de higiene y limpieza.

III Sólo puede imponerse por faltas graves y reiteración de
infracciones menos graves.


a) Sólo I

b) Sólo II

c) Sólo III

d) I y II

e) Todas son correctas

\begin{itemize}
\item Respuesta:

d), la ultima opción no es cierta ni precisa en su términos dado que
en estricto rigor solamente procede respecto de faltas graves, otra
cosa es que dentro de una de las hipótesis de falta grave sea la de
cometer 3 faltas menos graves en un bimestre.
\end{itemize}


En relación al abono del tiempo de privación de libertad en causa
diversa de aquella en que se originó la condena, señale la alternativa
correcta:


I La Corte Suprema ha resuelto la obligatoriedad del abono de la
privación de libertad sufrida en causas distintas a aquélla en que se
origina la condena, pero sugiriendo al mismo tiempo que sería
requisito para ello algún tipo de conexión de carácter temporal.

II Sólo se puede abonar el tiempo de detención y prisión preventiva,
pero no procede hacerlo respecto del arresto domiciliario.

III Sólo es procedente abonar el tiempo de privación de libertad
sufrido en una causa terminada en absolución.


a) Sólo I

b) I y II

c) Todas son correctas

d) Ninguna es correcta

e) Sólo III

\begin{itemize}
\item Respuesta:

a)
\end{itemize}


Durante un allanamiento se le encuentra a un condenado un teléfono
celular en su poder y se le aplica una sanción de privación de visita
por 3 semanasEl interno registraba sanciones disciplinarias
previas¿Cuál de las siguientes afirmaciones es correcta?

I Gendarmería, antes de proceder a la aplicación de la sanción, deberá
solicitar autorización al Juez de Garantía del lugar de reclusión por
tratarse de la repetición de una sanción disciplinaria.

II Copia de la resolución que sanciona la falta debe remitirse al
Director Regional de Gendarmería, quien podrá modificarla o anularla
por razones fundadas.

III En razón de la reincidencia del condenado es posible la
acumulación de la sanción impuesta con otra de menor entidad.


a) Sólo I

b) I y II

c) Todas son correctas

d) II y III

e) Sólo II

\begin{itemize}
\item Respuesta:

b)
\end{itemize}


¿Qué se entiende por “comportamiento sobresaliente” para los efectos
del beneficio de reducción de condena?

a) Haber sido calificado el condenado con nota “muy bueno” o “bueno”,
en los tres bimestres anteriores a su calificación.

b) No haber sido sancionado por faltas disciplinarias en los tres
bimestres anteriores.

c) No haber sido sancionado y participar en las actividades de
reinserción social durante el período que se califica.

d) Aquel que revelare notoria disposición del condenado para
participar positivamente en la vida social y comunitaria, una vez
terminada su condena.

e) Ninguna de las anteriores.


\begin{itemize}
\item Respuesta:

d)
\end{itemize}


Son requisitos para la libertad condicional:

I Estar condenado a una pena privativa de libertad de más de un año de
duración.

II Haber satisfecho la indemnización civil, si corresponde.

III Haber asistido con regularidad y provecho a la escuela y a las
conferencias educativas que se dicten, entendiéndose que no cumple con
este requisito el que no sabe leer y escribir.

a) Sólo I

b) I y II

c) I y III

d) Todas son correctas

e) Sólo III

\begin{itemize}
\item Respuesta:

c)
\end{itemize}


En relación con la regulación de las visitas que hace el Reglamento de
establecimientos penitenciarios, es correcto señalar:


I En las visitas ordinarias los menores de edad a los que se permite
ingresar al establecimiento deben tener más de catorce años.

II Los Alcaides pueden autorizar visitas familiares e íntimas pero
sólo a los internos que no gocen de permisos de salida.

III Los Jefes de los establecimientos pueden impedir las visitas de
personas cuya presentación sea indecorosa, claramente desaseada o
alterada.


a) Todas con correctas

b) Sólo I

c) I y II

d) I y III

e) Ninguna es correcta

\begin{itemize}
\item Respuesta:

a)
\end{itemize}


En relación con los permisos de salida regulados en el Reglamento de
Establecimientos Penitenciarios, señale cuál o cuáles son las
afirmaciones correctas.

I Para que un interno pueda postular a cualquier permiso de salida,
debe haber cumplido la mitad de la pena.

II La concesión, suspensión o revocación de los permisos de salida es
una facultad del Consejo Técnico.

III Salvo para el caso de la salida esporádica, para el resto de los
permisos de salida, se requiere haber observado muy buena conducta en
los tres bimestres anteriores a su postulación.

a) Todas son correctas

b) I y II son correctas

c) Sólo II y III son correctas

d) Sólo III es correcta

e) Ninguna es correcta

\begin{itemize}
\item Respuesta:

d)
\end{itemize}


Respecto de la posibilidad de interrumpir una pena privativa de
libertad y remplazarla por el régimen de libertad vigilada intensiva,
señale la alternativa correcta:


I La sanción impuesta al condenado no puede ser mayor a cinco años de
presidio o reclusión menor en su grado máximo.

II Para acceder a la pena mixta es necesario que el penado haya
cumplido al menos un tercio de la pena privativa de libertad de manera
efectiva.

III Los adolescentes condenados a internación en régimen cerrado con
programa de reinserción social no pueden acceder a la pena mixta.


a) Sólo I.


b) I y II.



c) Sólo II.


d) II y III.



e) Sólo III.


\begin{itemize}
\item Respuesta:

c)
\end{itemize}



En relación con las personas condenadas presidio perpetuo simple, es
correcto señalar que:


I Podrán postular a la salida dominical cuando hubieren cumplido 19
años de su pena, siempre que concurran los demás requisitos
establecidos en el Reglamento de Establecimientos Penitenciarios.

II Están excluidos del beneficio de reducción de condena.

III Podrán acceder a la salida esporádica, siempre que cumplan con las
condiciones establecidas en el Reglamento de Establecimientos
Penitenciarios.


a) Sólo I

b) I y II.


c) I y III son correctas.



d) Sólo II es correcta.


e) Todas son correctas.


\begin{itemize}
\item Respuesta:

e)
\end{itemize}


Luciano Tirado fue condenado a la pena de 61 días de presidio menor en
su grado mínimo, sustituyéndose la pena por la remisión condicional de
la misma por el lapso de 1 añoCuatro meses después de iniciar el
cumplimiento de dicha pena sustitutiva, el SrTirado es condenado por
haber cometido un nuevo simple delito¿Cuáles de las siguientes
afirmaciones es correcta en el caso del Sr Tirado?

I Procede la revocación de la pena sustitutiva y el abono de 20 días
por el tiempo que cumplió, por lo que el SrTirado sólo deberá cumplir
41 días de pena (sin perjuicio de la pena que se le impuso por el
nuevo delito).



II Se entenderá quebrantada la pena, por lo que le será revocada la
remisión condicional, y a la pena de 61 días de reclusión parcial se
debe abonar el tiempo el tiempo cumplido bajo remisión condicional,
que debe ser determinado prudencialmente por el tribunal.


III No corresponde el abono del tiempo cumplido bajo la modalidad de
remisión condicional y debe cumplir la pena privativa de libertad de
manera íntegra.


IV Declarado el quebrantamiento de la remisión de la pena, el tribunal
debe determinar si la pena debe ser cumplida de manera efectiva o bajo
la modalidad de libertad vigilada.


a) Sólo I.


b) I y IV.



c) Sólo II.



d) Sólo II y IV.



e) Sólo III


\begin{itemize}
\item Respuesta:

a), se funda en el siguiente razonamiento: el drTirado cumplió, en
proporciones, 1/3 de la pena sustitutiva (4 meses de 1
\end{itemize}
año), por lo que se le abona 20 días, es decir 1/3 de la pena
  primitiva de 61 días, quedando por cumplir 61 - 20 = 41 días.


En cuanto a la aplicación del artículo 164 del Código Orgánico de
Tribunales (COT), que regula la circunstancia de dictarse distintas
sentencias condenatorias contra un mismo imputado, es correcto
señalar:


I El tribunal competente es aquel tribunal que dictó la sentencia
posterior.

II Según la jurisprudencia de la Corte Suprema, ambos procesos deben
haber coincidido en el tiempo, de modo que la sentencia del primer
caso debe causar ejecutoria después que se haya iniciado la
investigación del segundo caso, y así sucesivamente en caso de
tratarse de varias condenas.

III Sólo es posible recurrir al Art.164 del COT cuando se trate de
delitos de la misma especie, pues así es aplicable el Art.351 del
Código Procesal Penal.

a) Sólo I es correcta

b) I y II son correctas

c) I y III son correctas

d) Todas son correctas

e) Sólo III es correcta

\begin{itemize}
\item Respuesta:

b)
\end{itemize}


Señale cuál o cuáles de las siguiente afirmaciones son correctas
respecto de los permisos de salida:

I Los permisos de salida son progresivos, por lo que necesariamente el
primer permiso al que se puede acceder es al de salida esporádica.

II Según el Reglamento de Establecimientos Penitenciarios, las
sesiones de los Consejos Técnicos serán secretas.

III Aunque un recluso haya observado muy buena conducta en los tres
bimestres anteriores a su postulación, de todas maneras el Reglamento
permite denegar el permiso de salida si, con anterioridad a los tres
bimestres referidos, registra infracciones disciplinarias graves que
así lo aconsejen.

a) Todas son correctas

b) Ninguna es correcta

c) Sólo I es correcta

d) I y II son correctas

e) II y III son correctas

\begin{itemize}
\item Respuesta:

e)
\end{itemize}


En relación a la facultad de Gendarmería para trasladar reclusos, es
acertado decir que:


I Es una sanción administrativa que sólo procede frente a reiteración
de faltas disciplinarias graves.

II Es una facultad del Jefe del Establecimiento.

III Es facultad de Gendarmería, y no corresponde a un tribunal la
decisión del lugar donde se debe cumplir una condena privativa de
libertad.

a) Sólo III

b) Sólo II

c) I y II

d) Todas son correctas.


e) II y III

\begin{itemize}
\item Respuesta:

a)
\end{itemize}


En relación al beneficio de reducción de condena es correcto señalar
lo siguiente:


I Para obtener el beneficio de reducción de condena es necesario que
el interesado postule ante el Jefe de Establecimiento respectivo.

II El beneficio de reducción de condena opera solo una vez en la vida
de una persona, de tal manera que si el condenado ya obtuvo este
beneficio respecto de una condena anterior queda excluido en relación
con otras condenas futuras.

III El condenado que ha incumplido las condiciones impuestas durante
el régimen de libertad condicional está excluido del beneficio de
reducción de condena.

IV En caso de que el establecimiento respectivo no disponga de
talleres escolares, no es posible exigir este requisito para que la
Comisión de Rebaja califique la conducta como sobresaliente.


a) Todas son correctas.


b) II, III y IV

c) I y II

d) I y IV

e) II y III

\begin{itemize}
\item Respuesta:

b)
\end{itemize}


En materia de control judicial de la ejecución de la pena, es correcto
señalar que:

I En nuestro ordenamiento jurídico ningún cuerpo legal contempla un
recurso que tenga por finalidad específica impugnar judicialmente una
resolución o decisión adoptada por la administración penitenciaria.

II La competencia en materia de ejecución de sentencias criminales y
medidas de seguridad está radicada en el juez de garantía del lugar
donde la pena deba cumplirse.

III No existe ninguna norma en el Reglamento de Establecimientos
Penitenciarios que establezca el control judicial permanente de la
ejecución de la pena.

a) Sólo I

b) I y II

c) Todas son correctas.


d) Sólo II

e) I y III

\begin{itemize}
\item Respuesta:

e)
\end{itemize}


Las siguientes afirmaciones son correctas en relación a la posición de
garante del Estado respecto de las personas privadas libertad:


I El Estado debe garantizar aquellos derechos que las personas
privadas de libertad no pueden satisfacerse a sí mismos, por su
particular situación de dependencia en relación con el órgano del
Estado que ejerce su custodia.

II Existe una presunción por la cual el Estado es responsable por las
lesiones que exhibe una persona que ha estado bajo la custodia de
agentes estatales.

III La privación de libertad trae a menudo, como consecuencia
ineludible, la afectación del goce de otros derechos humanos además
del derecho a la libertad personal; pueden, por ejemplo, verse
restringidos los derechos de privacidad y de intimidad familiarEsta
restricción de derechos, consecuencia de la privación de libertad o
efecto colateral de la misma, sin embargo, debe limitarse de manera
rigurosa.


a) Sólo I

b) I y III

c) Todas son correctas

d) I y II

e) Ninguna es correcta

\begin{itemize}
\item Respuesta:

c)
\end{itemize}


Respecto de la pena mixta, es correcto señalar que:


I No es procedente si se trata del cómplice del delito consumado de
sustracción de menores, a menos que se le reconozca en la sentencia la
circunstancia primera del artículo 11 del Código Penal.

II Sí es procedente tratándose del autor del delito consumado de
parricidio, siempre que se le hubiere reconocido en la sentencia la
circunstancia primera del artículo 11 del Código Penal.

III No procede respecto del cómplice de violación impropia, a menos
que se le reconozca en la sentencia la circunstancia primera del
artículo 11 del Código Penal.

a) Todas con correctas

b) II y III

c) I y II

d) Sólo II

e) Ninguna es correcta

\begin{itemize}
\item Respuesta:

d)
\end{itemize}


En relación con la pena mixta es correcto afirmar que:


I Según la Ley 18.216 la pena mixta debe ser siempre controlada
mediante monitoreo telemático.


II Si se dispone la interrupción de la pena privativa de libertad, el
plazo de observación de la libertad vigilada intensiva que debe fijar
el tribunal es igual al de duración de la pena que al condenado le
restare por cumplir.

III Si el tribunal no otorga la pena mixta, ésta no puede discutirse
sino hasta transcurridos seis meses desde su denegación.

a) Todas son correctas

b) Ninguna es correcta

c) Sólo II es correcta

d) I y III son correctas

e) II y III son correctas

\begin{itemize}
\item Respuesta:

a)
\end{itemize}


¿Cuál o cuáles de las siguientes afirmaciones son correctas respecto
de la sanción de internación en celda solitaria?:


I Según el Reglamento de Establecimientos Penitenciarios, la
aplicación de esta sanción implica necesariamente una rebaja en la
calificación de la conducta.

II Respecto de las mujeres que tuvieren hijos consigo sólo puede
aplicarse esta sanción por un máximo de cinco días.

III Los internos sancionados con esta medida deberán ser visitados
diariamente por el Jefe del Establecimiento.

a) Sólo III es correcta

b) I y II son correctas

c) I y III son correctas

d) II y III son correctas

e) Todas son correctas

\begin{itemize}
\item Respuesta:

c)
\end{itemize}


Según la Ley 18.216, en caso de revocación de una pena sustitutiva se
debe abonar al cumplimiento efectivo el tiempo de ejecución de dicha
pena de forma proporcional. Esto significa que:


I En caso de reclusión parcial, se abona 1 día por cada 24 horas de
reclusión.

II Teniendo en cuenta que la remisión condicional es menos rigurosa
que la reclusión parcial, en el caso de la primera se abona 1 día por
cada 2.

III Queda a criterio del juez el abono a aplicar, caso en el cuál
puede solicitar informe a Gendarmería.

IV Sólo se aplica para el caso en que la pena sustitutiva no sea
idéntica a la condena.

a) Todas son correctas

b) Sólo IV

c) I, II y IV

d) I, III y IV

e) I y IV

\begin{itemize}
\item Respuesta:

b)
\end{itemize}


Para la adopción de una sanción en forma ajustada a la falta
disciplinaria, se considerarán, además de la gravedad de la misma:


I La conducta del interno dentro del año.

II El tiempo de cumplimiento de la condena.

III Las características del interno.


a) Todas son correctas

b) Sólo I es correcta

c) Sólo II es correcta

d) Sólo III es correcta

e) I y III son correctas

\begin{itemize}
\item Respuesta:

e)
\end{itemize}


En relación con la libertad condicional es cierto que:

I Todas las personas condenadas a más de un año de privación de
libertad tienen derecho a solicitar la libertad condicional,
independiente del delito cometido y de la pena impuesta, siempre que
cumplan los demás requisitos legales.

II El requisito de “conducta intachable” en el establecimiento penal,
según el DL 321, implica completar dos bimestres con conducta muy
buena o sobresaliente.

III La libertad condicional es otorgada por Decreto del Ministro de
Justicia a proposición de la Comisión de libertad condicional.

a) Sólo I

b) I y II

c) Sólo II

d) II y III

e) Todas son correctas

\begin{itemize}
\item Respuesta:

a)
\end{itemize}


Sobre la pena sustitutiva de expulsión, es correcto señalar que:

I Según la Ley N° 18.216, el tribunal puede de oficio sustituir el
cumplimiento de la pena impuesta por la expulsión del extranjero
condenado.

II No procede respecto del autor de tráfico ilícito de
estupefacientes.

III Generalmente se decreta la expulsión, pues administrativamente y
de acuerdo a lo dispuesto en el DL 1094 (Ley de Extranjería), el
extranjero condenado, será deportado en todo caso.


a) Sólo I

b) Sólo II

c) Todas son correctas

d) I y II

e) I y III

\begin{itemize}
\item Respuesta:

a), en realidad la respuesta correcta actualmente es la d), dado que
con la ultima modificación legislativa, año 2021, se vetó la
posibilidad de sustitución respecto de infracción a ley 20.000.
\end{itemize}


En relación con la regulación de las visitas que hace el Reglamento de
establecimientos penitenciarios, es correcto señalar:


I En las visitas ordinarias cada interno pude ser visitado por un
máximo de 5 personas simultáneamente.

II En las visitas ordinarias los menores de edad deberán tener más de
catorce años.

III Las visitas especiales (familiares e íntimas) se conceden a
internos que no gocen de permisos de salida.



a) I y II

b) II y III

c) I y III

d) Sólo II

e) Todas son correctas

\begin{itemize}
\item Respuesta:

e)
\end{itemize}


Uno de los requisitos que deben reunir los condenados para postular a
casi todos los permisos de salida es poseer “Muy buena Conducta”Esta
calificación se le exigirá por el lapso, inmediatamente anterior a la
postulación, de:

I Un bimestre.

II Dos bimestres.

III Tres bimestres.

IV Cuatro bimestres.

V Seis bimestres.


a) Sólo I

b) Sólo II

c) Sólo III

d) Sólo IV

e) Sólo V

\begin{itemize}
\item Respuesta:

c)
\end{itemize}


Las siguientes aseveraciones son correctas en relación con el abono de
tiempo de privación de libertad en causa diversa:


I La jurisprudencia de la Corte Suprema ha señalado que al abono se
aplica la regla del artículo 164 del Código Orgánico de Tribunales, es
decir, la exigencia de que ambas causas (la de la privación de
libertad que se abona y la de la condena en que se abona el tiempo)
hayan podido juzgarse conjuntamente.


II Es procedente abonar a una condena el tiempo de prisión preventiva
que sufrió una persona en causa diversa cuando esta última causa
terminó en sobreseimiento definitivo.

III Es procedente abonar a una condena el tiempo de prisión preventiva
que sufrió una persona en causa diversa cuando esta última causa
terminó por aplicación del principio de oportunidad, pero no se puede
abonar el tiempo de detención o de ampliación de la misma.


a) Todas son correctas

b) I y II son correctas

c) Ninguna es correcta

d) Sólo II es correcta

e) II y III son correctas

\begin{itemize}
\item Respuesta:

b)
\end{itemize}


\subsection{Verdadero y Falso}
\label{sec:org54a8cc5}

Para acceder al beneficio de libertad condicional no se requiere
necesariamente haber obtenido y cumplido satisfactoriamente algún tipo de
permiso de salida.

V

Por la acción de revisión, y para el caso en que no se reúnan todas las
condiciones que permitan la anulación de la condena, se puede pedir la
rebaja de la pena de manera proporcional

F

El condenado extranjero que ha sido objeto de expulsión de todas maneras
puede obtener permiso de salida.

V

Toda sanción por alguna falta disciplinaria debe ser aplicada por el Jefe del
Establecimiento.

V

La sanción de internación en celda solitaria sólo puede imponerse por faltas
graves y reiteración de infracciones menos graves.

F

La salida controlada al medio libre debe autorizarse con vigilancia.

F

En nuestro sistema penitenciario el juez encargado de hacer ejecutar las
condenas criminales y las medidas de seguridad, y resolver las solicitudes y
reclamos relativos a dicha ejecución es el juez que dictó la sentencia o impuso
la medida de seguridad.

F

Los condenados por hurto o estafa a más seis años, pueden obtener la libertad
condicional una vez cumplidos tres años.

V

El desaseo en la presentación personal constituye una falta disciplinaria en el
Reglamento de Establecimientos Penitenciarios.

V

Los permisos de salida no proceden en caso de condenas privativas de libertad
inferiores a un año.

F

El indulto particular no procede respecto de condenados que hubieren sido
indultados anteriormente.

V

\begin{enumerate}
\item La decisión que se adopte a este respecto se la solicitud de pena mixta no es
\end{enumerate}
apelable

F

El condenado beneficiado con la pena mixta no podrá acceder al remplazo de la
libertad vigilada intensiva por libertad vigilada.

V

En caso de quebrantamiento o incumplimiento voluntario de las condiciones de
algún permiso, al reingreso, el interno tendrá una conducta calificada con la nota
mínima.

V

\begin{enumerate}
\item Las personas condenadas a penas privativas de libertad iguales o menores a un
\end{enumerate}
año no pueden postular a la libertad condicional.

V

Los Alcaides pueden autorizar visitas familiares e íntimas pero sólo a los internos
que no gocen de permisos de salida.

V

El traslado de un condenado a otra Unidad Penal es una sanción administrativa
que sólo procede respecto de una infracción grave.

F

En nuestro ordenamiento jurídico ningún cuerpo legal contempla un recurso que
tenga por finalidad específica impugnar judicialmente una resolución o decisión
adoptada por la administración penitenciaria.

V

La competencia en materia de ejecución de sentencias criminales y medidas de
seguridad está radicada en el juez de garantía del lugar donde la pena deba
cumplirse.

F

El condenado que ha incumplido las condiciones impuestas durante el régimen de
libertad condicional está excluido del beneficio de reducción de condena.

V

La revocación de la pena sustitutiva de prestación de servicios en beneficio de la
comunidad trae como consecuencia el cumplimiento íntegro de la pena privativa
de libertad originalmente impuesta.

F

Para efectos de la libertad condicional, se entiende por “tiempo de condena” el
total de las condenas que tenga el reo, con excepción de las que se impongan
mientras se cumplen éstas.

F

Según el Reglamento de Establecimientos Penitenciarios, el desaseo de un
condenado en su presentación personal constituye una falta disciplinaria.

V

La Ley N° 19.880, que establece bases de los procedimientos administrativos que
rigen los actos de la administración del Estado, es aplicable supletoriamente a los
procedimientos de Gendarmería de Chile.

V

No procede el beneficio de reducción de condena respecto de los condenados que
hubiere quebrantado su condena, a menos que mejore sustancialmente su
conducta, siendo calificada como muy buena durante los dos semestres
posteriores al reinicio del cumplimiento de su pena.

F

Según el Reglamento de Establecimientos Penitenciarios, la facultad de aplicar
sanciones disciplinarias a los internos está radicada en el Jefe del Establecimiento,
no obstante, la repetición de toda medida disciplinaria a un condenado debe ser
autorizada por el Juez del lugar de reclusión.

V

\begin{enumerate}
\item Para acceder a la salida dominical es necesario haber sido beneficiado previamente
\end{enumerate}
de la salida esporádica y haber dado cumplimiento cabal a la totalidad de las
obligaciones que impone esta última salida.

F

El condenado por homicidio calificado (Art. 391 N° 1 del Código Penal), al que no le
reconoce ninguna circunstancia modificatoria de responsabilidad penal, no podrá
optar al beneficio de reducción de su condena aunque la pena impuesta sea
inferior a presidio perpetuo.

V

Antes de otorgarse un permiso de salida a un extranjero condenado debe
recabarse si el interno tiene o no orden de expulsión del país.

V

Los condenados pueden rehusarse a participar en la programación de las
actividades y acciones para la reinserción social sin consecuencias disciplinarias.

V

La salida controlada al medio libre sólo puede otorgarse con autorización judicial.

F

Los extranjeros que ingresen a nuestro país en calidad de refugiados o asilados
políticos sólo podrán ser expulsados administrativamente si es que son
condenados por crimen.

F

Constituye circunstancia agravante cometer el delito durante el tiempo
correspondiente al período condonado en virtud del beneficio de reducción de
condena.

V

En nuestro sistema penitenciario el juez encargado de hacer ejecutar las condenas
criminales y las medidas de seguridad, y resolver las solicitudes y reclamos
relativos a dicha ejecución es el juez que dictó la sentencia o impuso la medida de
seguridad.

F

Es posible autorizar a internos para que realicen trabajos fuera del recinto
penitenciario pero, sólo podrá ser en beneficio de la comunidad o encontrarse
justificado en relación a algún programa de rehabilitación, capacitación o empleo.

V

En las audiencias judiciales relativas a determinar la situación del condenado que
cae en enajenación mental debe intervenir el Ministerio Público, el defensor y el
Curador ad litem, si existe.

V

La salida controlada al medio libre debe sólo puede autorizarse con sistema de
vigilancia electrónico (brazalete).

F

El tiempo que un condenado hubiere permanecido en prisión preventiva durante
todo o parte del respectivo proceso se computará para los efectos de proceder a la
calificación de comportamiento sobresaliente del beneficio de reducción de
condena.

V

La Ley de N° 19.856 permite postular al beneficio de reducción de condena a
aquellas personas que se encontraren cumpliendo una condena en régimen de
reclusión nocturna.

V

En nuestra normativa penitenciaria así como en el derecho internacional de los
derechos humanos, el fin atribuido a las penas privativas de libertad durante la
etapa de su ejecución es la prevención especial positiva.

V


Los condenados por hurto o estafa a más de seis años, podrán obtener
la libertad condicional una vez cumplidos tres años.


\begin{itemize}
\item Respuesta:
\end{itemize}

V

La decisión que niega la libertad condicional de una persona condenada
es apelable ante la respectiva Corte de Apelaciones.


\begin{itemize}
\item Respuesta:
\end{itemize}

F, se presenta amparo.



El traslado de personas condenadas a otros módulos o establecimientos
es una sanción que sólo puede aplicarse a faltas graves.


\begin{itemize}
\item Respuesta:
\end{itemize}

F


El beneficio de reducción de condena no tendrá lugar si el condenado
hubiere cometido algún delito al que la ley asigna como pena máxima el
presidio perpetuo, salvo en los casos en que en la determinación de la
pena se hubiera considerado la circunstancia primera establecida en el
artículo del Código Penal.


\begin{itemize}
\item Respuesta:
\end{itemize}

F


Según el Reglamento de establecimientos Penitenciarios, el Jefe del
Establecimiento puede denegar un permiso de salida aun cuando el
informe del Consejo Técnico sea favorable.


\begin{itemize}
\item Respuesta:
\end{itemize}

V

El informe psicosocial de Gendarmería de Chile que contenga una
opinión técnica favorable que permita orientar sobre los factores de
riesgo de reincidencia, a fin de conocer las posibilidades del
condenado para reinsertarse adecuadamente en la sociedad es requisito
fundamental para la obtención de la libertad condicional.


\begin{itemize}
\item Respuesta:
\end{itemize}

F


Constituirá circunstancia agravante, cometer el delito durante el
tiempo correspondiente al período condonado en virtud del beneficio de
reducción de condena.


\begin{itemize}
\item Respuesta:
\end{itemize}

V

Quienes hubieren demostrado comportamiento sobresaliente según los
procesos de calificación para la obtención del beneficio de reducción
de condena, estarán habilitados para postular al régimen de libertad
condicional en el semestre anterior a aquel en que les hubiere
correspondido.


\begin{itemize}
\item Respuesta:
\end{itemize}

V


Sin perjuicio de las demás reglas especiales, a los condenados a más
de veinte años se les podrá conceder el beneficio de la libertad
condicional una vez cumplidos diez años de la pena, y por este solo
hecho ésta quedará fijada en veinte años.


\begin{itemize}
\item Respuesta:
\end{itemize}

V

El Reglamento de Establecimientos Penitenciarios no permite imponer
sanción de internación en celda solitaria a mujeres.


\begin{itemize}
\item Respuesta:
\end{itemize}

F, solamente no permite respecto a mujeres en determinadas
circunstancias ahí descritas.



Los condenados a penas iguales o inferiores a un año de privación de
libertad no tienen derecho a postular a la libertad condicional.


\begin{itemize}
\item Respuesta:
\end{itemize}

V


En caso de concederse la pena mixta, el tribunal fijará el plazo de
observación de la libertad vigilada intensiva por un período igual o
superior al de duración de la pena que al condenado le restare por
cumplir, con un límite máximo de cinco años.


\begin{itemize}
\item Respuesta:
\end{itemize}

F



Habiéndose decretado la revocación de la pena sustitutiva de
prestación de servicios en beneficio de la comunidad, se abonará al
tiempo de reclusión un día por cada ocho horas efectivamente
trabajadas.


\begin{itemize}
\item Respuesta:
\end{itemize}

V


El condenado extranjero que ha sido objeto de expulsión administrativa
sólo puede acceder al permiso de salida esporádica en casos
calificados.


\begin{itemize}
\item Respuesta:
\end{itemize}

F



Según el Reglamento de Establecimientos Penitenciarios los internos
tendrán derecho a efectuar peticiones a las autoridades
penitenciarias, las que deben ser respondidas por el alcaide en el
plazo de quince días corridos o, a lo menos, dentro del mismo plazo,
deberá informarse el estado de tramitación en que se encuentra.


\begin{itemize}
\item Respuesta:
\end{itemize}

V


La Comisión de Libertad Condicional funciona en la Corte de
Apelaciones respectiva, durante los meses de abril y octubre de cada
año.


\begin{itemize}
\item Respuesta:
\end{itemize}

V

Para rechazar un permiso de salida controlada al medio libre de un
interno que ha hecho uso provechoso de los permisos de salida
dominical y de fin de semana, el Jefe de Establecimiento debe contar
con acuerdo del Consejo Técnico.


\begin{itemize}
\item Respuesta:
\end{itemize}

F


La cesación del comportamiento sobresaliente en un período de
calificación, importará la pérdida completa de las reducciones de
condena correspondientes a los años precedentes, sin perjuicio de la
procedencia futura del beneficio en el evento de que el condenado
retomare el comportamiento sobresaliente exigido y la subsistencia de
hasta un 80\% del beneficio de reducción de condena acumulado si se
autoriza de conformidad a la ley.


\begin{itemize}
\item Respuesta:
\end{itemize}

V

El Jefe del Establecimiento podrá autorizar la salida, con las
condiciones establecidas en el Reglamento de Establecimientos
Penitenciarios, de los internos que habiendo cumplido un tercio de su
pena privativa de libertad hayan sido propuestos por el Consejo
Técnico como merecedores de este permiso como premio o estímulo
especial.


\begin{itemize}
\item Respuesta:
\end{itemize}

V

Para efectos del beneficio de reducción de condena, el tiempo que un
condenado hubiere permanecido en prisión preventiva durante todo o
parte del respectivo proceso, será calificado una vez impuesta la
sentencia condenatoria, en el marco del primer período anual de
calificación.


\begin{itemize}
\item Respuesta:
\end{itemize}

V


La persona privada de libertad puede rehusarse a participar en la
programación de las actividades de reinserción social sin que ello le
reporte consecuencias disciplinarias.

V F

\begin{itemize}
\item Respuesta:

V
\end{itemize}

En las audiencias judiciales relativas a determinar la situación del
condenado que cae en enajenación mental no se admite al querellante
como interviniente.


V F

\begin{itemize}
\item Respuesta:

V
\end{itemize}

Un problema serio de nuestro ordenamiento jurídico en materia
penitenciaria es que la Ley 19.880, que establece bases de los
procedimientos administrativos que rigen los actos de la
administración del Estado, no es aplicable supletoriamente a los
procedimientos de Gendarmería de Chile.


V F

\begin{itemize}
\item Respuesta:

F
\end{itemize}


Constituye circunstancia agravante cometer el delito durante el tiempo
correspondiente al período condonado en virtud del beneficio de
reducción de condena.


V F

\begin{itemize}
\item Respuesta:

V
\end{itemize}

En nuestro sistema penitenciario el juez encargado de hacer ejecutar
las condenas criminales y las medidas de seguridad, y resolver las
solicitudes y reclamos relativos a dicha ejecución es el juez que
dictó la sentencia o impuso la medida de seguridad.


V F

\begin{itemize}
\item Respuesta:

F, es el JG
\end{itemize}


Los permisos de salida no proceden en caso de condenas privativas de
libertad inferiores a un año.


V F

\begin{itemize}
\item Respuesta:

F
\end{itemize}


Los condenados por hurto o estafa a más seis años, pueden obtener la
libertad condicional una vez cumplidos tres años.


V F

\begin{itemize}
\item Respuesta:

V
\end{itemize}

Los extranjeros que ingresen a nuestro país en calidad de refugiados o
asilados políticos sólo podrán ser expulsados si es que son condenados
por crimen.


V F

\begin{itemize}
\item Respuesta:

F
\end{itemize}


En el derecho internacional de los derechos humanos, los diversos
pactos y declaraciones adscriben a un sistema preventivo especial
positivo en el cumplimiento de las penas.


V F

\begin{itemize}
\item Respuesta:

V
\end{itemize}

La salida controlada al medio libre debe autorizarse con vigilancia.


V F

\begin{itemize}
\item Respuesta:

F
\end{itemize}


Siendo uno de los requisitos de procedencia que el condenado haya sido
evaluado con conducta “buena” o “muy buena”, durante los 3 bimestres
anteriores a la solicitud, la pena mínima exigible para postular a la
pena mixta es precisamente de 6 meses.


V F


\begin{itemize}
\item Respuesta:

V
\end{itemize}

En materia de reducción de condena, se entiende por “tiempo de
condena” el total de las penas que tenga el interno, incluyendo las
que se impongan mientras cumple éstas, deducidas las rebajas que
hubiere obtenido.


V F

\begin{itemize}
\item Respuesta:

V
\end{itemize}

La comisión de tres faltas menos graves durante un semestre constituye
falta grave.

V F

\begin{itemize}
\item Respuesta:

F
\end{itemize}


Para aplicar una sanción administrativa, el Reglamento de
Establecimientos Penitenciarios exige que debe oírse al interno sólo
en caso de una infracción grave.

V F

\begin{itemize}
\item Respuesta:

V
\end{itemize}

En las audiencias judiciales relativas a determinar la situación del
condenado que cae en enajenación mental no se admite la intervención
del querellante.

V F

\begin{itemize}
\item Respuesta:

V
\end{itemize}

Por tratarse de una especie de libertad vigilada, la pena mínima que
habilita a postular a pena mixta es de 3 años por regla general y 541
días en caso de ciertos delitos, todo ello de acuerdo al artículo 15
bis de la Ley 18.216.

V F

\begin{itemize}
\item Respuesta:

F, el artseñala como mínimo respecto de determinados delitos 540 d

y no 541
\end{itemize}

La repetición de toda medida disciplinaria debe comunicarse al Juez de
Garantía que intervino en el respectivo proceso penal.

V F

\begin{itemize}
\item Respuesta:

F, se trata del Juez del lugar de reclusión art87.
\end{itemize}



El condenado extranjero no puede obtener permiso de salida si ha sido
objeto de expulsión.

V F

\begin{itemize}
\item Respuesta:

F
\end{itemize}


Alterar el descanso de los demás internos en cualquier forma,
constituye una falta disciplinaria.

V F

\begin{itemize}
\item Respuesta:

V
\end{itemize}

La salida controlada al medio libre sólo puede otorgarse con el objeto
de desarrollar actividades laborales o académicas.

V F

\begin{itemize}
\item Respuesta:

V
\end{itemize}

Para acceder al beneficio de libertad condicional es necesario haber
obtenido y cumplido satisfactoriamente algún tipo de permiso de
salida.

V F

\begin{itemize}
\item Respuesta:

F
\end{itemize}


Constituye circunstancia agravante cometer el delito durante el tiempo
correspondiente al período condonado en virtud del beneficio de
reducción de condena.

V F

\begin{itemize}
\item Respuesta:

V
\end{itemize}

No procede el beneficio de reducción de condena respecto del condenado
que hubiere delinquido durante el proceso, estando sujeto a alguna
medida cautelar distinta de la prisión preventiva.

V F

\begin{itemize}
\item Respuesta:

V
\end{itemize}

Los actos de Gendarmería de Chile son actos administrativosSin
embargo, como cuentan con una normativa particular, no son aplicables
las normas de la Ley N° 19.880, que establece bases de los
procedimientos administrativos que rigen los actos de la
administración del Estado.

V F

\begin{itemize}
\item Respuesta:

F
\end{itemize}


Según el Reglamento de Establecimientos Penitenciarios, las faltas
disciplinarias que pueden constituir delito no pueden sancionarse
administrativamente mientras no se pronuncie respecto de los hechos la
justicia penal.

V F

\begin{itemize}
\item Respuesta:

F
\end{itemize}


En materia de beneficio de reducción de condena, la cesación del
comportamiento sobresaliente en un período de calificación, importa,
en principio, la pérdida completa de las reducciones de condena
correspondientes a los años precedentes.

V F

\begin{itemize}
\item Respuesta:

V
\end{itemize}

El tiempo que un condenado hubiere permanecido en prisión preventiva
durante todo o parte del respectivo proceso, no se computará para los
efectos de proceder a la calificación de conducta sobresaliente
necesaria para el beneficio de reducción de condena.

V F

\begin{itemize}
\item Respuesta:

F
\end{itemize}


Los Jefes de los establecimientos podrán autorizar visitas íntimas una
vez al mes, si las condiciones del establecimiento lo permiten, y
siempre que él o la privado (a) de libertad acredite que quien lo
visita es su cónyuge.


V F

\begin{itemize}
\item Respuesta:

F
\end{itemize}


Los Jefes de turno al interior del establecimiento pueden disponer la
incomunicación o aislamiento provisorio de cualquier interno que
incurriere en falta grave, dando cuenta de inmediato al Jefe del
Establecimiento.

V F

\begin{itemize}
\item Respuesta:

V
\end{itemize}

Por la acción de revisión, y para el caso en que no se reúnan todas
las condiciones que permitan la anulación de la condena, se puede
pedir la rebaja de la pena de manera proporcional.

V F

\begin{itemize}
\item Respuesta:

F
\end{itemize}
\end{document}